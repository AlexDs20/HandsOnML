\section{The machine learning landscape}
%----------------------------------------
ML is nothing more than fitting a model to the known data!
\subsection{Sturcture of a project}
\begin{enumerate}
    \item study the problem
    \item Train the ML algo (with lots of data)
    \item Solution
    \item Check the solution and better understand the problem
    \item[$\rightarrow$] try to improve model from the observed possible problems
\end{enumerate}

Examples:
\begin{itemize}
    \item Detection of signals in images using CNNs
    \item Classifying articles: NLP
\end{itemize}


\subsection{Types of ML}
Different categories of ML algorithms:

\begin{tabular}{p{0.20\textwidth}p{0.40\textwidth}p{0.4\textwidth}}
  \textbf{Properties} & \textbf{Descrition} & \textbf{Examples} \\
  \hline
  Supervised &  Need laballed training data & kNN, Linear Reg., Logistic Reg., SVM, Decision Tree, Random Forest, NN\ldots \\
  Unsupervised & No labelled data, the algorithm puts similar data together & Clustering: k-means, DBSCAN, HCA. Dimensional reduction: PCA, kernel PCA, LLE, t-SNE. Anomaly detection: One-class SVM, Isolation Forest\\
  Semi-Supervised & Both labelled and not labelled data & Deepd belief network, restricted Boltzman machine \\
  Reinforcment Learning & In the learning process give rewards or penalites depending on the action done & \\
  \hline\hline
  Batch Learning & Trains using ALL data $\rightarrow$ If needs to retrain, need to train on all the data again & \\
  Online Learning & Train by mini-batches: Train incrementally and can start again from a previous minibatch run & \\
  \hline\hline
  Instance Based & On new data, check distance to known data and assign same output &\\
  Model-based & Use the data to make predictions / interpolates to new data & \\
\end{tabular}

\subsubsection{Unsupervised Learning}
\begin{description}
    \item[Feature Extraction] Combines related features into a better one (e.g. using PCA)
    \item[Anomaly Detection] Find anomalies in dataset(e.g.\ removing some outliers)
    \item[Association rule learning] Discover relationships in large datasets
\end{description}

\subsubsection{semi-supervised Learning}
Often much data but only a few are labelled
\begin{itemize}
  \item[$\rightarrow$] Often combinaison of supervised and unsupervised algo.
\end{itemize}

\subsubsection{Reinforcement Learning}
Trains an agent by giving rewards and penalties to obtained the best policy

\subsection{Challenges}
\subsubsection{data}
\begin{description}
  \item[Lack of data] ML requires lots of data (mininmum thousands of examples).\\
                      Note that models performances increase with number of data.\\
                      $\rightarrow$ Choice to work on model or gather more data!

  \item[Non representative data] Data used for training must include similar data to those for predictions. (Poor extrapolation capacity)

  \item[Sampling Bias] non representative data because the sampling is flawed.\\
          Bias can appear if the proportion of data in different classes are not representative.

  \item[Poor data quality] if errors, outliers and noise in data

  \item[Irrelevant features] There should be enough relevant features and not much crap.
\end{description}


\subsubsection{Fitting}
\begin{description}
  \item[Overfitting] The model fits too well the training data. Occurs because model too complex compared to data $\rightarrow$ lose predictability.
  \item[Underfitting] Model too simple compared to the data to be represented e.g.\ Linear fit on a non-linear problem
\end{description}

Solutions to overfitting:
\begin{itemize}
  \item \textit{Regularization}\\
    $\rightarrow$ Making the model simpler to avoid by adjusting hyperparameter
  \item \textit{Hyperparameter}\\
    $\rightarrow$ Parameters of the learning algorithm that dictates the amount of regularization.
\end{itemize}

Solutions to underfitting
\begin{itemize}
  \item Select a better model
  \item Have better features
  \item reduce the constraints on the model (e.g.\ hyperparameters)
\end{itemize}

\subsection{Evaluating performance}

Split the data into training set (80\%) and testing set (20\%).
Evaluation on the test set gives an estimate of the error on unseen data.
When training multiple models on the training set and testing them on the test sets, our selection of the "best" model is which model best fits the test data set.
$\rightarrow$ Model cannot necessarily be good on new data.
Thus, keep a validation set.

Thus, there are 3 sets of data: Training, Validation and Test
\begin{enumerate}
  \item Train multiple models with different Hyperparameter on the training set
  \item Select the model that performs best on the validation set.
  \item Verify that the model is indeed good on the test set
  \item Deploy
\end{enumerate}

It is common to split the training set to train several models on sub-sets of the training set and train one final model on the whole training set.

If the validation set is too small $\rightarrow$ imprecise evaluation of performances.
The validation set cannot be too large either (compared to training set).\\
$\rightarrow$ use \emph{cross-validation}: it uses several small validation sets.
Each model is evaluated once per validation set.
You can then average the results of the model on the different smaller validation sets.

\subsection{Data mismatch}
If there are two types of data in the training set (e.g.\ not from the same source).
This may cause problem in the predictions.
How to know if overfitting or problem with the data?

Hold part of the training set (data from one of the source) and see how the model performs on that.
If it performs well $\rightarrow$ the error comes from the mismatched data
If it performs poorly $\rightarrow$ the error comes from the overfitting

If it is because of the data mismatched, is it possible to preprocess these to be more like the rest of the data?
