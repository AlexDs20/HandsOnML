\newpage
\section{Decision Trees}

    Can perform classification, regression and multiple output tasks.\\

    They are the fundamentals of Random Forests (among the most powerful algo.).\\

    Decision Trees require little data preparation.
    It does not require feature scaling or centering.\\

\subsection{Training and visualizing}

\begin{python}
    data = load_data()
    X = data.data
    y = data.target
    model = DecisionTreeClassifier(max_depth=2)
    model.fit(X, y)
\end{python}

The tree can be be exported (and then visualized after conversion) using
\begin{python}
    export_graphviz(
        model,
        out_file="output_file.dot"
        feature_names=data.feature_names,
        class_names=data.target_names,
        rounded=True,
        filled=True
    )
\end{python}
and can be converted in a shell using:
\begin{python}
    dot -Tpng file_name.dot -o file_name.png
\end{python}

\subsubsection*{Reading tree}
In the output:\\
\begin{itemize}
    \item first line = Condition to be satisfied
    \item gini = measures the impurity of the node (pure if gini=0)
            It is pure if all the training instances satisfying the condition belong to the same class.
    \item samples = how many samples are subject to the condition check (first line)
    \item value = [.., .., .., ..] = How many training instances of this node applied to
    \item class = Predicted class
\end{itemize}

\subsection{Making predictions}

The gini score for the $i^\mathrm{th}$ node:
\begin{equation}
    G_i = 1 - \sum_{k=1}^{n}p_{i,k}^2
\end{equation}
with $p_{i,k}$ the ratio of class k instances among the training instances in the $i^\mathrm{th}$ node.\\

It can estimate class and probability.\\
However, the probability is determine by the region (hyper-rectangle) in which it belongs to in the feature space depending on the amount of class of each instances in the hyper-rectangle.

\subsection{Making predictions}

Scikit-learn uses the Classification and Regression (CART) algorithm to train decision trees and it only allow to split each node into two categories.\\

There exists other algorithms (ID3) which allow to split into more children categories.\\

\subsubsection*{The CART algorithm}
Split the training set into sub-set by setting threshold $t_k$ on the feature $k$.\\
Which pair $\left(k, t_k\right)$? $\rightarrow$ produces the purest subsets.

Cost function:
\begin{equation}
    J\left(k, tk\right) = \frac{m_\mathrm{left}}{m} G_\mathrm{left} + \frac{m_\mathrm{right}}{m} G_\mathrm{right}
\end{equation}
with $G_{l/r}$ measures the impurity of the left/right subset and $m_{l/r}$ the number of instances in the left/right subset.

This method gets itterated at each nodes until reaching the desired depth or if it cannot reduce the impurity.

\subsubsection*{Gini impurity or Entropy}
sklearn $\rightarrow$ Gini impurity (default) but entropy can also be chosen exist.\\

\begin{itemize}
\item thermo.:\\
Entropy $\rightarrow$ measures disorder.

\item Shannon's information theory:\\
It measures the average information content of a message.\\
0 = all messages are the same.

\item Machine learning: $\rightarrow$ measures impurity.
0 = if set contains instances of only one class.\\
Definition: $H_i = -\sum_{k=1}^n p_{i,k} log_2(p_{i,k})$
\end{itemize}

Gini or entropy?\\
Usually, no big difference.
\begin{description}
    \item [Gini] \hfill\\
    \begin{itemize}
        \item Faster to computer $\rightarrow$ good default\\
        \item Tends to isolate the most frequent class in its own branch
    \end{itemize}
    \item [Entropy] \hfill\\
    Produces more balanced trees.
\end{description}

\subsubsection*{Regularization hyperparameters}
Decision tree does not assume anything about the data and just fit the training data.
$\rightarrow$ Often overfitting.\\

"non-parametric" model because the number of parameters is not set before training.
To not overfit $\rightarrow$ limitation during training = "Regularization".
\begin{itemize}
    \item \verb;max_depth;
    \item \verb;min_samples_split;: min number of sample a node must have before it can split
    \item \verb;min_samples_leaf;: minimum number of samples a leaf must have
    \item \verb;min_weight_fraction_leaf;: min fraction of instances (compared to total instances) a leaf must have
    \item \verb;max_leaf_nodes;: max number of leaf nodes
    \item \verb;max_features;
\end{itemize}
$\Rightarrow$ regularize the model

\subsection{Regression}
Trees can also be used for regressions.\\
Following the fitted tree result in predicting values.\\
Instead of minimizing the impurity (as in classification), it minimizes the "MSE".\\
Again, this is prone to overfitting and regularization should be used (e.g.\ \verb;min_samples_leaf;).

\subsection{Limitations}
\begin{description}
    \item [Perpendicular splits] \hfill\\
    A training set rotated by $45^\circ$ in the feature space might very well, or badly be splitted using decision tree.\\
    To limit this: $\Rightarrow$ use Principal Component Analysis.
    \item [Sensitivity] \hfill\\
    Sensitive to small changes in the training set.\\
    Random Forests can limit this by averaging predictions from many trees.
\end{description}
